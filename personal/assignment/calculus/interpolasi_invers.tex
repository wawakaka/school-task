\documentclass{article}
\usepackage[utf8]{inputenc}
\usepackage{amsmath, amssymb}

\title{Interpolasi Invers dan Peran Monotonik}
\author{}
\date{}

\begin{document}

\maketitle

\section*{1. Apa arti monotonik dalam hal ini?}

Dalam konteks interpolasi invers, \textbf{monotonik} berarti fungsi yang menghubungkan nilai $x$ dan $y$ bersifat \textit{selalu naik} (monoton naik) atau \textit{selalu turun} (monoton turun) pada interval yang ditinjau.
\begin{itemize}
    \item \textbf{Monoton Naik}: Jika $x_1 < x_2$, maka $f(x_1) \leq f(x_2)$ (atau secara ketat: $f(x_1) < f(x_2)$).
    \item \textbf{Monoton Turun}: Jika $x_1 < x_2$, maka $f(x_1) \geq f(x_2)$ (atau secara ketat: $f(x_1) > f(x_2)$).
\end{itemize}

Artinya, tidak ada ``naik-turun'' atau fluktuasi — nilai fungsi hanya bergerak satu arah seiring bertambahnya input.

\section*{2. Mengapa monotonik menjadi masalah dalam interpolasi invers?}

Interpolasi invers adalah proses mencari nilai $x$ yang sesuai dengan nilai $y$ tertentu, berdasarkan pasangan data $(x_i, y_i)$. Masalah muncul jika fungsi \textbf{tidak monotonik}, karena:

\subsection*{a. Tidak Unik (Non-Unique Solution)}
Jika fungsi tidak monotonik, maka satu nilai $y$ dapat berkorespondensi dengan lebih dari satu nilai $x$.

\textbf{Contoh:} Misalkan $y = x^2$ pada interval $[-2, 2]$. Untuk $y = 4$, terdapat dua solusi: $x = -2$ dan $x = 2$. \\
$\Rightarrow$ Interpolasi invers tidak tahu solusi mana yang harus dipilih.

\subsection*{b. Tidak Invertibel (Not Invertible)}
Fungsi yang tidak monotonik \textbf{tidak memiliki invers yang merupakan fungsi}, karena melanggar definisi fungsi (satu input $\rightarrow$ satu output). Interpolasi invers mengasumsikan bahwa hubungan antara $x$ dan $y$ dapat dibalik, yang hanya dijamin jika fungsi monotonik.

\subsection*{c. Kesalahan Interpolasi}
Algoritma interpolasi (misalnya Lagrange atau spline) dapat memberikan hasil yang salah atau tidak stabil jika diterapkan pada data non-monotonik, karena tidak jelas arah pencarian nilai $x$ untuk $y$ tertentu.

\section*{Kesimpulan}

\begin{description}
    \item[$\checkmark$] Interpolasi invers bekerja dengan baik hanya jika fungsi \textbf{monotonik}, karena:
    \item Menjamin keunikan solusi.
        \item Memastikan fungsi invertibel.
        \item Mencegah ambiguitas dan kesalahan numerik.
    
    \item[$\times$] Jika fungsi tidak monotonik, pertimbangkan:
    \item Membagi interval menjadi bagian-bagian yang monotonik.
        \item Menggunakan metode alternatif seperti pencarian akar (\textit{root finding}) dengan batasan tambahan.
\end{description}

\end{document}