\documentclass[12pt]{article}
\usepackage{amsmath}
\usepackage{amssymb}
\usepackage{geometry}
\geometry{margin=1in}

\title{}
\author{}
\date{}

\begin{document}

\section*{Sistem Persamaan Linear}
\[
\begin{cases}
2x + y - 2z = -1 \\
3x - 3y - z = 5 \\
x - 2y + 3z = 6
\end{cases}
\]

\section*{Bentuk Augmented Matrix}
\[
\left[\begin{array}{ccc|c}
2 & 1 & -2 & -1 \\
3 & -3 & -1 & 5 \\
1 & -2 & 3 & 6
\end{array}\right]
\]

\section*{Operasi Baris Elementer (OBE)}

\subsection*{Tukar baris untuk memudahkan pivot}
Tukar \( R_1 \leftrightarrow R_3 \):
\[
\left[\begin{array}{ccc|c}
1 & -2 & 3 & 6 \\
3 & -3 & -1 & 5 \\
2 & 1 & -2 & -1
\end{array}\right]
\]

\subsection*{Nolkan elemen di bawah pivot kolom pertama}
\[
\begin{aligned}
R_2 &\leftarrow R_2 - 3R_1 \\
R_3 &\leftarrow R_3 - 2R_1
\end{aligned}
\quad \Rightarrow \quad
\left[\begin{array}{ccc|c}
1 & -2 & 3 & 6 \\
0 & 3 & -10 & -13 \\
0 & 5 & -8 & -13
\end{array}\right]
\]

\subsection*{Buat pivot baris kedua menjadi 1}
\[
R_2 \leftarrow \frac{1}{3}R_2 \quad \Rightarrow \quad
\left[\begin{array}{ccc|c}
1 & -2 & 3 & 6 \\
0 & 1 & -\frac{10}{3} & -\frac{13}{3} \\
0 & 5 & -8 & -13
\end{array}\right]
\]

\subsection*{Nolkan elemen di bawah pivot kolom kedua}
\[
R_3 \leftarrow R_3 - 5R_2 \quad \Rightarrow \quad
\left[\begin{array}{ccc|c}
1 & -2 & 3 & 6 \\
0 & 1 & -\frac{10}{3} & -\frac{13}{3} \\
0 & 0 & \frac{26}{3} & \frac{26}{3}
\end{array}\right]
\]

\subsection*{Buat pivot baris ketiga menjadi 1}
\[
R_3 \leftarrow \frac{3}{26}R_3 \quad \Rightarrow \quad
\left[\begin{array}{ccc|c}
1 & -2 & 3 & 6 \\
0 & 1 & -\frac{10}{3} & -\frac{13}{3} \\
0 & 0 & 1 & 1
\end{array}\right]
\]

Matriks ini sudah dalam bentuk eselon baris.

\section*{Ubah kembali ke sistem persamaan}
\[
\begin{cases}
x - 2y + 3z = 6 \\
y - \dfrac{10}{3}z = -\dfrac{13}{3} \\
z = 1
\end{cases}
\]

\section*{Substitusi Mundur}

Dari persamaan ketiga: \( z = 1 \).

Substitusi ke persamaan kedua:
\[
y - \frac{10}{3}(1) = -\frac{13}{3} \quad \Rightarrow \quad y = -1
\]

Substitusi \( y = -1 \) dan \( z = 1 \) ke persamaan pertama:
\[
x - 2(-1) + 3(1) = 6 \quad \Rightarrow \quad x + 2 + 3 = 6 \quad \Rightarrow \quad x = 1
\]

\section*{Solusi Akhir}
\[
\boxed{
\begin{aligned}
x &= 1 \\
y &= -1 \\
z &= 1
\end{aligned}
}
\]

\section*{Verifikasi}
\begin{itemize}
    \item Persamaan 1: \( 2(1) + (-1) - 2(1) = -1 \) 
    \item Persamaan 2: \( 3(1) - 3(-1) - 1 = 5 \) 
    \item Persamaan 3: \( 1 - 2(-1) + 3(1) = 6 \) 
\end{itemize}


\end{document}