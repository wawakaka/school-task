\documentclass{article}
\usepackage{amsmath}
\usepackage{amssymb}
\usepackage{array}
\author{}
\date{}

\begin{document}

Diberikan sistem persamaan linear:
\[
\begin{cases}
2x + y + z = 9 \\
x + 2y - z = 6 \\
3x - y + 2z = 17
\end{cases}
\]

Bentuk matriks augmented dari sistem tersebut adalah:
\[
\left[\begin{array}{ccc|c}
2 & 1 & 1 & 9 \\
1 & 2 & -1 & 6 \\
3 & -1 & 2 & 17
\end{array}\right]
\]

Kita akan melakukan operasi baris elementer untuk mengubah matriks ini menjadi bentuk eselon baris tereduksi (reduced row echelon form).

\textbf{Langkah 1:} Tukar baris 1 dan baris 2 agar elemen pivot di kolom pertama bernilai 1.
\[
R_1 \leftrightarrow R_2
\quad \Rightarrow \quad
\left[\begin{array}{ccc|c}
1 & 2 & -1 & 6 \\
2 & 1 & 1 & 9 \\
3 & -1 & 2 & 17
\end{array}\right]
\]

\textbf{Langkah 2:} Hilangkan elemen di bawah pivot baris pertama.
\[
R_2 \leftarrow R_2 - 2R_1, \quad R_3 \leftarrow R_3 - 3R_1
\]
\[
\left[\begin{array}{ccc|c}
1 & 2 & -1 & 6 \\
0 & -3 & 3 & -3 \\
0 & -7 & 5 & -1
\end{array}\right]
\]

\textbf{Langkah 3:} Buat pivot baris kedua menjadi 1 dengan membagi baris kedua dengan $-3$.
\[
R_2 \leftarrow -\frac{1}{3} R_2
\quad \Rightarrow \quad
\left[\begin{array}{ccc|c}
1 & 2 & -1 & 6 \\
0 & 1 & -1 & 1 \\
0 & -7 & 5 & -1
\end{array}\right]
\]

\textbf{Langkah 4:} Hilangkan elemen di atas dan di bawah pivot baris kedua.
\[
R_1 \leftarrow R_1 - 2R_2, \quad R_3 \leftarrow R_3 + 7R_2
\]
\[
\left[\begin{array}{ccc|c}
1 & 0 & 1 & 4 \\
0 & 1 & -1 & 1 \\
0 & 0 & -2 & 6
\end{array}\right]
\]

\textbf{Langkah 5:} Buat pivot baris ketiga menjadi 1.
\[
R_3 \leftarrow -\frac{1}{2} R_3
\quad \Rightarrow \quad
\left[\begin{array}{ccc|c}
1 & 0 & 1 & 4 \\
0 & 1 & -1 & 1 \\
0 & 0 & 1 & -3
\end{array}\right]
\]

\textbf{Langkah 6:} Hilangkan elemen di atas pivot baris ketiga.
\[
R_1 \leftarrow R_1 - R_3, \quad R_2 \leftarrow R_2 + R_3
\]
\[
\left[\begin{array}{ccc|c}
1 & 0 & 0 & 7 \\
0 & 1 & 0 & -2 \\
0 & 0 & 1 & -3
\end{array}\right]
\]

Matriks sekarang berada dalam bentuk eselon baris tereduksi. Dari sini, kita dapat membaca solusi langsung:

\[
x = 7, \quad y = -2, \quad z = -3
\]

\textbf{Jadi, solusi dari sistem persamaan linear tersebut adalah:}
\[
\boxed{x = 7, \quad y = -2, \quad z = -3}
\]

\end{document}